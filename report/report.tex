\documentclass[14pt,a4paper,final]{extreport}
\usepackage[utf8]{inputenc}
\usepackage[main=ukrainian,english]{babel}
\usepackage[left=2cm,top=2cm,right=1cm,bottom=2cm]{geometry}
\usepackage{graphicx}
\usepackage{titlesec}
\usepackage{verbatim}
\usepackage{float}
\usepackage{listings}
\usepackage{placeins}
\usepackage{hyperref}

\graphicspath{{img/}}
\setkeys{Gin}{width=\columnwidth}
\usepackage{indentfirst}
\usepackage{minted}
\setminted{breaklines}

% Some custom commands
\newcommand{\img}[2]{
\begin{figure}[H]
\includegraphics{#1}
#2
\end{figure}
}

% usage template
% \img{load_conf}{\caption{Завантаження конфігурації}}
% \section*{TODO}
% \addcontentsline{toc}{section}{TODO}

\title{Комп'ютерний практикум №1}
\date{}

\begin{document}
\begin{center}
    Міністерство освіти і науки України

    Національний технічний університет України

    "Київський політехнічний інститут імені Ігоря Сікорського"

    Фізико-технічний інститут
\end{center}
\vspace{\baselineskip}
\vspace{\baselineskip}
\vspace{\baselineskip}
\begin{center}
    Дисципліна "Теоретико-числовi алгоритми в криптологiї"
\end{center}
    \vspace{\baselineskip}
    \begin{center}
        Комп'ютерний практикум №2. Застосування алгоритму дискретного логарифмування
    \end{center}
\vspace{\baselineskip}
\vspace{\baselineskip}
\vspace{\baselineskip}
\begin{center}
    Виконали
\end{center}
\begin{flushright}
    студент гр. ФБ-11 Подолянко Т.О.
\end{flushright}
\vfill
\begin{center}
    Київ — 2024
\end{center}
\thispagestyle{empty}
\pagebreak

Мета роботи: ознайомитися з алгоритмом Сiльвера-Полiга-Геллмана пошуку дискретного логарифму, практично реалізувати.
Дослідити переваги та недоліки даного алгоритму. Практично оцінити складність роботи алгоритму.


\section*{Постановка задачі}

Програмно реалізувати алгоритм пошуку дискретного логарифму Сiльвера-Полiга-Геллмана; пошук дискретного логарифму шляхом звичайного перебору.
Створити Docker image реалізованої програми. За допомогою задач, створених генератором, оцінити і порівняти ефективність реалізованих алгоритмів, 
створити візуалізацію.

\section*{Хід роботи}

Для програмної реалізації обрано мову Python. Початкова реалізація використовувала пакет numpy, але цей пакет не містить достатнього функціоналу
для роботи з модульною арифметикою. Це, і той факт, що numpy оперує числами фіксованої довжини до 64 біт, призвело до можливості розв'язувати задачі
з довжиною простого модуля лише до 7 десяткових чисел (інакше відбувалося переповнення цілого). Тому довелося вже використаний функціонал numpy
реалізувати на чистому Python, з аналогічним інтерфейсом, що потенційно могло знизити продуктивність.

Для факторизації порядку групи застосовано програму, реалізовану у першій лабораторній роботі.

Для автоматизації збирання статистики про виконання було реалізовано функціонал взаємодії з генератором задач, автоматичного розв'язку та збереження 
статистик у форматі CSV (тип задачі, довжина $p$, $a$, $b$, $p$, розв'язок $x$, час пошуку у розв'язку у ns).

Реалізовано інтерфейс командного рядку для застосунку для демонстрації та тестування.

Вихідний код реалізованої програми розміщено на платформі GitHUb за посиланням \href{https://github.com/timofey282228/nta-lab2}{https://github.com/timofey282228/nta-lab2}.

Docker-контейнер з реалізованою програмою доступний за посиланням \url{https://hub.docker.com/repository/docker/timofey282228/nta-lab2}. 
Для встановлення Docker-контейнеру та роботи з програмою достатньо виконати такі команди:

\begin{minted}{bash}
docker pull timofey282228/nta-lab2
# Тепер образ контейнера доступний локально і з ним можна працювати
# Виведемо всі опції
docker run --rm -it timofey282228/nta-lab2 --help
\end{minted}

\section*{Результати дослідження}

Задачі мають вид $x = \log_a{b}$. У випадку перевищення часу виконання у 5хв
програма завершується з повідомленням \verb|ERROR:WATCHDOG:Timed out|.

Задачі першого типу мають порядки, канонічний розклад яких не містить
великих простих чисел. Натомість задачі другого типу мають у канонічному
розкладі порядку великі прості.

Приклад запуску програми для розв'язання задачі пошуку дискретного логарифма за алгоритмом прямого перебору:

\begin{lstlisting}
python -m nta_lab2 --bruteforce benchmark
    Solving tasks of length 3
    Task type 1:
     a = 44; b = 345; p = 349.
     Solution: x = 280
    
    Task type 2:
     a = 6; b = 258; p = 733.
     Solution: x = 102
    
    Solving tasks of length 4
    Task type 1:
     a = 3890; b = 1921; p = 4231.
     Solution: x = 1236
    
    Task type 2:
     a = 1758; b = 4153; p = 8423.
     Solution: x = 903
    
    Solving tasks of length 5
    Task type 1:
     a = 32212; b = 94720; p = 97841.
     Solution: x = 95135
    
    Task type 2:
     a = 4600; b = 57354; p = 64489.
     Solution: x = 6378
    
    Solving tasks of length 6
    Task type 1:
     a = 427594; b = 264491; p = 690661.
     Solution: x = 345968
    
    Task type 2:
     a = 298083; b = 257987; p = 348923.
     Solution: x = 342940
    
    Solving tasks of length 7
    Task type 1:
     a = 1172927; b = 2782538; p = 4347823.
     Solution: x = 2093505
    
    Task type 2:
     a = 3068031; b = 165082; p = 4399709.
     Solution: x = 1826848
    
    Solving tasks of length 8
    Task type 1:
     a = 19154660; b = 13782820; p = 51462143.
     Solution: x = 30604010
    
    Task type 2:
     a = 2129439; b = 78825779; p = 86496131.
     Solution: x = 61464803
    
    Solving tasks of length 9
    Task type 1:
     a = 87804180; b = 156091201; p = 157900361.
     Solution: x = 143988912
    
    Task type 2:
     a = 624422518; b = 178537248; p = 796517357.
    ERROR:WATCHDOG:Timed out
    
\end{lstlisting}

Приклад запуску програми для розв'язання задачі дискретного логарифма за алгоритмом С-П-Г:

\begin{lstlisting}
python -m nta_lab2 --sph benchmark
    Solving tasks of length 3
    Task type 1:
     a = 456; b = 459; p = 997.
     Solution: x = 61
    
    Task type 2:
     a = 7; b = 156; p = 349.
     Solution: x = 233
    
    Solving tasks of length 4
    Task type 1:
     a = 623; b = 308; p = 2287.
     Solution: x = 1732
    
    Task type 2:
     a = 3449; b = 696; p = 9887.
     Solution: x = 3837
    
    Solving tasks of length 5
    Task type 1:
     a = 20914; b = 10413; p = 32831.
     Solution: x = 29391
    
    Task type 2:
     a = 68999; b = 26111; p = 84499.
     Solution: x = 18562
    
    Solving tasks of length 6
    Task type 1:
     a = 414654; b = 18515; p = 494383.
     Solution: x = 443546
    
    Task type 2:
     a = 728487; b = 468981; p = 794579.
     Solution: x = 94339
    
    Solving tasks of length 7
    Task type 1:
     a = 1783464; b = 480104; p = 4102649.
     Solution: x = 114070
    
    Task type 2:
     a = 529939; b = 4870138; p = 5974961.
     Solution: x = 2516083
    
    Solving tasks of length 8
    Task type 1:
     a = 28328173; b = 42675221; p = 85737073.
     Solution: x = 48541524
    
    Task type 2:
     a = 67737035; b = 49848676; p = 79464113.
     Solution: x = 48733402
    
    Solving tasks of length 9
    Task type 1:
     a = 59766419; b = 211314191; p = 242998309.
     Solution: x = 58537437
    
    Task type 2:
     a = 515691475; b = 177550495; p = 635225827.
     Solution: x = 568602013
    
    Solving tasks of length 10
    Task type 1:
     a = 3613561375; b = 5818654269; p = 7177166771.
     Solution: x = 4643599436
    
    Task type 2:
     a = 6520939515; b = 2829697272; p = 7588291267.
     Solution: x = 3073832271
    
    Solving tasks of length 11
    Task type 1:
     a = 42828629342; b = 65892885923; p = 74773661353.
     Solution: x = 64422358455
    
    Task type 2:
     a = 13599806083; b = 3893794467; p = 34654422953.
     Solution: x = 29870963672
    
    Solving tasks of length 12
    Task type 1:
     a = 265594761127; b = 210038288446; p = 269290148401.
     Solution: x = 55943295488
    
    Task type 2:
     a = 646446651524; b = 44545113074; p = 918385787329.
     Solution: x = 799327359889
    
    Solving tasks of length 13
    Task type 1:
     a = 1533944836144; b = 967742581498; p = 1639261586423.
     Solution: x = 949015360787
    
    Task type 2:
     a = 1104903456453; b = 645427471193; p = 1924766750761.
     Solution: x = 1336203121828
    
    Solving tasks of length 14
    Task type 1:
     a = 11856322708936; b = 31677183208601; p = 36372688880959.
     Solution: x = 1853264609151
    
    Task type 2:
     a = 44859063828363; b = 21307356532290; p = 95470442482589.
    ERROR:WATCHDOG:Timed out
    
\end{lstlisting}

\section*{Продуктивність роботи програмної реалізації}

\img{sph_1}

\img{sph_2}

\img{brute_1}

\img{brute_2}

\section*{Оцінка максимального порядку вхідного параметра $p$}

Практично встановлено, що максимальний порядок параметра $p$, за якого задачу
дискретного логарифмування можна розв'язати за допомогою даної
реалізації алгоритму С-П-Г за відведений час (5хв), дорівнює 13.

\section*{Висновки}

Алгоритм Сiльвера-Полiга-Геллмана дозволяє знайти дискретний логарифм
елемента мультиплікативної групи з поліноміальною складністю відносно порядку
порядку. Якщо порядок групи в канонічному розкладі
містить великі прості числа чи прості числа великого степеня, то 
очікувана складність алгоритму експоненційна.
Для групи простого порядку алгоритм вироджується у повний перебір.



\end{document}